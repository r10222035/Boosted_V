%!TEX program = xelatex
%!TEX options=--shell-escape
\documentclass[12pt]{article}

%
\usepackage[scheme=plain]{ctex}
%
\usepackage{fontspec}
%
\usepackage[margin = 1in]{geometry}

%
\usepackage[dvipsnames]{xcolor}
\usepackage[many]{tcolorbox}

%
\usepackage{amsmath}
\usepackage{amssymb}
\usepackage{amsthm}
%
\usepackage{tensor}
%
\usepackage{slashed}
\usepackage{physics}
\usepackage{simpler-wick}

%
\usepackage[version=4]{mhchem}

%
\usepackage{mathtools}

%
\usepackage{bm}
\newcommand{\dbar}{\dif\hspace*{-0.18em}\bar{}\hspace*{0.2em}}
\DeclareMathAlphabet\mathbfcal{OMS}{cmsy}{b}{n}
%\usepackage{bbold}
\newcommand*{\dif}{\mathop{}\!\mathrm{d}}
\newcommand*{\euler}{\mathrm{e}}
\newcommand*{\imagi}{\mathrm{i}}

\renewcommand{\vec}[1]{\boldsymbol{\mathbf{#1}}}

\usepackage{caption}

\usepackage{enumitem}

\usepackage{multirow}
%
\usepackage{mathrsfs}
\usepackage{dsfont}

%
\usepackage{hyperref}
\hypersetup{
    colorlinks=true,
    linkcolor=violet,
    filecolor=blue,      
    urlcolor=blue,
    citecolor=cyan,
}

%
\usepackage{graphicx}
\usepackage{subfig}
\usepackage{subcaption}
%
\graphicspath{{figures/}}


%
\usepackage{indentfirst}
%
\setlength{\parindent}{2em}
\linespread{1.25}

% 
% \setmainfont{Times New Roman}

\title{Note}
\author{Feng-Yang Hsieh}
\date{}

\begin{document}
\maketitle

\section{Sample}% (fold)
\label{sec:sample}
	The vector boson samples are generated from \textsc{MadGraph5 aMC@NLO 3.3.1}. Then pass to Pythia for showering and hadronization. Then pass to Delphes for detector simulation. 	

	The processes are the VBF production for doubly charged Higgses $H_5^{\pm\pm}$ and heavy neutral Higgses $H_5$ with decays $H_5^{\pm\pm} \to W^{\pm}W^{\pm} \to jjjj$ and $H_5\to ZZ \to jjjj$.

	Following are the MadGraph scripts for generating $W^{+}$ sample:
	\begin{verbatim}
	import model GM_NLO
	define v = z w+ w-
	generate p p > H5pp j j $$v, (H5pp > w+ w+, (w+ > j j), (w+ > j j))
	output VBF_H5pp_ww_jjjj
	launch VBF_H5pp_ww_jjjj

	shower=Pythia8
	detector=Delphes
	analysis=OFF
	done

	set param_card tanth 0.226480
	set param_card lam2 0.070070
	set param_card lam3 -1.331328
	set param_card lam4 1.364671
	set param_card lam5 -1.963271
	set param_card M1coeff 1046.827111
	set param_card M2coeff 135.30791

	set run_card nevents 100000
	set run_card ebeam1 6500.0
	set run_card ebeam2 6500.0
	set run_card cut_decays True
	set run_card ptj 50
	set run_card etaj 3

	/home/r10222035/boosted_V_ML_test/Cards/delphes_card.dat

	done	
	\end{verbatim}

	In Delphes, jets are constructed by the anti-$k_t$ algorithm with $R = 0.7$.

	\subsection{Sample selection}% (fold)
	\label{sub:sample_selection}
		Only the events that satisfy the following requirement will use for training.
		\begin{enumerate}
			\item The transverse momentum of jets are required $p_\text{T} \in (350, 450) \text{ GeV}$ and in range $\abs{\eta}<1$.
			\item Merging: The angular distance between the two quarks decayed from the vector boson is required $\Delta R(q_1,q_2) < 0.6$.
			\item Matching: The vector boson and jet are matched if $\Delta R(V,j) < 0.1$. The events with less than two matching jets will be discarded.
		\end{enumerate}
	% subsection sample_selection (end)
	
	\subsection{Sample pre-processing}% (fold)
	\label{sub:sample_pre_processing}
		The jet charge of a jet is defined as
		\begin{equation}
			\mathcal{Q}_\kappa \equiv \frac{1}{(p_{\text{T}, J})^{\kappa}} \sum_{i\in J} q_i \times \left( p_{\text{T}}^{i} \right)^{\kappa}
		\end{equation}
		where $i$ represents the constituent in the jet $J$.

		\subsubsection{Jet image}% (fold)
		\label{subs:jet_image}
			After the sample selection, we can construct the jet image of the matching vector boson jet. The jet image is constructed by the following steps:
			\begin{enumerate}
				\item Centralization: Calculating the $p_\text{T}$ weighted center in $\eta,\phi$ plane, then shift this point to origin.
				\item Rotation
				\item Flipping
				\item Pixelating in $\Delta\eta = \Delta\phi = 1.6$ box, with $75\times 75$ pixels.
			\end{enumerate}
		% subsubsection jet_image (end)	
	% subsection sample_pre_processing (end)
	\subsection{Plots}% (fold)
	\label{sub:plots}
		Figure \ref{fig:jet_mass_distribution} is the jet mass distribution.
		\begin{figure}[htpb]
			\centering
			\includegraphics[width=0.5\textwidth]{jet_mass_distribution.png}
			\caption{Jet mass distributions of their vector boson samples.}
			\label{fig:jet_mass_distribution}
		\end{figure}

		Figure \ref{fig:jet_charge_different_kappa} is the jet charge distributions with different $\kappa$.
		\begin{figure}[htpb]
			\centering
			\includegraphics[width=0.95\textwidth]{jet_charge_distribution.png}
			\caption{$\mathcal{Q}_\kappa$ distributions of vector boson samples.}
			\label{fig:jet_charge_different_kappa}
		\end{figure}

		Figure \ref{fig:jet_image_PT}, \ref{fig:jet_image_Qk} are the average jet images of $p_\text{T}$ and $\mathcal{Q}_\kappa$.
		\begin{figure}[htpb]
			\centering
			\includegraphics[width=0.95\textwidth]{jet_image_PT.png}
			\caption{Average of jet images in the $p_\text{T}$ channel.}
			\label{fig:jet_image_PT}
		\end{figure}
		\begin{figure}[htpb]
			\centering
			\includegraphics[width=0.95\textwidth]{jet_image_Qk.png}
			\caption{Average of jet images in the $\mathcal{Q}_\kappa$ channel, with $\kappa = 0.15$.}
			\label{fig:jet_image_Qk}
		\end{figure}
		Figure \ref{fig:jet_image_PT_Z-W+} is the $Z$ jet image minus $W^{+}$ jet image in $p_\text{T}$ channel.
		\begin{figure}[htpb]
			\centering
			\includegraphics[width=0.5\textwidth]{jet_image_PT_Z-W+.png}
			\caption{The difference between the $Z$ and $W^{+}$ average jet images in $p_\text{T}$ channel.}
			\label{fig:jet_image_PT_Z-W+}
		\end{figure}

	% subsection plots (end)		

% section sample (end)
\section{BDT}% (fold)
\label{sec:bdt}
	The input features are jet mass $\mathcal{M}$ and jet charge $\mathcal{Q}_\kappa$.

	\subsection{BDT training}% (fold)
	\label{sub:bdt_training}
		Use the GBDT implemented by scikit-learn. The parameters are set as the default setting in sklearn.

		The training and testing sample size are in Table \ref{tab:BDT_sample_size}.
		\begin{table}[htpb]
			\centering
			\caption{The entries in the sum correspond to the $(W^{+}, W^{-}, Z)$ samples.}
			\label{tab:BDT_sample_size}
			\begin{tabular}{c|c|c}
			Case & Training set    & Testing set   \\ \hline
			1    & $14k+15k+13k$   & $3k+3k+3k$    \\
			2    & $100k+105k+90k$ & $25k+26k+22k$ \\
			3    & $242k+257k+221k$& $60k+64k+54k$ \\
			\end{tabular}
		\end{table}
	% subsection bdt_training (end)

	\subsection{BDT results}% (fold)
	\label{sub:bdt_results}
		The training results are summarized in Table \ref{tab:BDT_training_result}. The ROC of BDT for $\kappa = 0.3$ is presented in Figure \ref{fig:ROC_of_BDT}. 
		\begin{table}[htpb]
			\centering
			\caption{The training results of BDT for different $\kappa$ samples.}
			\label{tab:BDT_training_result}
			\begin{tabular}{c|c|c|cc|cc|cc}
								  &					 &             & \multicolumn{2}{c|}{$W^{+}$} & \multicolumn{2}{c|}{$W^{-}$} & \multicolumn{2}{c}{$Z$} \\
								  & Sample			 & Overall ACC & AUC		  & ACC			 & AUC          & ACC		   & AUC        & ACC       \\ \hline
			    BDT $\kappa=0.15$ &\multirow{2}{*}{1}& 0.644 & 0.808 & 0.773 & 0.812 & 0.766 & 0.812 & 0.770 \\ 
				BDT $\kappa=0.30$ &                  & 0.649 & 0.822 & 0.778 & 0.822 & 0.769 & 0.814 & 0.773 \\ \hline
			    BDT $\kappa=0.15$ &\multirow{2}{*}{2}& 0.646 & 0.808 & 0.769 & 0.815 & 0.765 & 0.809 & 0.774 \\ 
			    BDT $\kappa=0.30$ &					 & 0.652 & 0.820 & 0.774 & 0.826 & 0.772 & 0.810 & 0.775 \\ \hline
			    BDT $\kappa=0.15$ &\multirow{2}{*}{3}& 0.648 & 0.811 & 0.769 & 0.817 & 0.767 & 0.810 & 0.775 \\ 
			    BDT $\kappa=0.30$ &					 & 0.654 & 0.824 & 0.776 & 0.829 & 0.773 & 0.811 & 0.774 \\
			\end{tabular}
		\end{table}

		\begin{figure}[htpb]
			\centering
			\includegraphics[width=0.8\textwidth]{ROC_BDT.png}
			\caption{The ROC of BDT. Case $W^{+}$ means take $W^{+}$ samples as signal and $W^{-}, Z$ as background.}
			\label{fig:ROC_of_BDT}
		\end{figure}

		\begin{figure}[htpb]
			\centering
			\includegraphics[width=0.9\textwidth]{figures/True_and_BDT_distribution_of_M_Qk.png}
			\caption{The left plot shows the true distributions for $W^{+}, W^{-}, Z$  in the $(\mathcal{Q}_\kappa ,\mathcal{M})$ plane (for $\kappa=0.3$). The right plot is the BDT prediction.}
			\label{fig:M_Qk_distribution}
		\end{figure}

		% subsection bdt_results (end)
% section bdt (end)	
\section{CNN}% (fold)
\label{sec:cnn}
	The inputs are the jet image of $p_T$ and $\mathcal{Q}_\kappa$.
	\subsection{CNN training}% (fold)
	\label{sub:cnn_training}
		The training, validation, and testing sample size are in Table \ref{tab:CNN_sample_size}.
		\begin{table}[htpb]
			\centering
			\caption{The entries in the sum correspond to the $(W^{+}, W^{-}, Z)$ samples.}
			\label{tab:CNN_sample_size}
			\begin{tabular}{c|c|c|c}
			Case & Training set     & Validation set & Testing set   \\ \hline
			1    & $11k+12k+10k$    & $2k+2k+2k$     & $3k+3k+3k$    \\
			2    & $80k+84k+72k$    & $19k+21k+18k$  & $25k+26k+22k$ \\
			3    & $194k+205k+176k$ & $48k+51k+44k$  & $60k+64k+55k$
			\end{tabular}
		\end{table}
	% subsection cnn_training (end)
	\subsection{CNN results}% (fold)
	\label{sub:cnn_results}
		The training results are summarized in Table \ref{tab:CNN_training_result}. The ROC of CNN for $\kappa = 0.15$ is presented in Figure \ref{fig:CNN_of_BDT}.
		\begin{table}[htpb]
			\centering
			\caption{The training results of CNN with different $\kappa$ samples.}
			\label{tab:CNN_training_result}
			\begin{tabular}{c|c|c|cc|cc|cc}
								  &					  &             & \multicolumn{2}{c|}{$W^{+}$} & \multicolumn{2}{c|}{$W^{-}$} & \multicolumn{2}{c}{$Z$} \\
								  & Sample			  & Overall ACC & AUC        & ACC       & AUC        & ACC       & AUC       & ACC       \\ \hline
				CNN $\kappa=0.15$ & \multirow{2}{*}{1}& 0.639 & 0.825 & 0.772 & 0.829 & 0.774 & 0.782 & 0.768 \\
				CNN $\kappa=0.30$ &					  & 0.632 & 0.822 & 0.771 & 0.825 & 0.771 & 0.786 & 0.771 \\ \hline
				CNN $\kappa=0.15$ & \multirow{2}{*}{2}& 0.667 & 0.847 & 0.790 & 0.847 & 0.784 & 0.828 & 0.795 \\
				CNN $\kappa=0.30$ &					  & 0.663 & 0.843 & 0.786 & 0.843 & 0.779 & 0.825 & 0.795 \\ \hline
				CNN $\kappa=0.15$ & \multirow{2}{*}{3}& 0.672 & 0.849 & 0.793 & 0.851 & 0.787 & 0.834 & 0.799 \\
				CNN $\kappa=0.30$ &					  & 0.666 & 0.845 & 0.788 & 0.847 & 0.783 & 0.832 & 0.800 \\
			\end{tabular}
		\end{table}	
		\begin{figure}[htpb]
			\centering
			\includegraphics[width=0.8\textwidth]{ROC_CNN.png}
			\caption{The ROC of CNN for $\kappa = 0.15$. Case $W^{+}$ means take $W^{+}$ samples as signal and $W^{-}, Z$ as background.}
			\label{fig:CNN_of_BDT}
		\end{figure}
		
	% subsection cnn_results (end)	
% section cnn (end)		
\section{CNN\texorpdfstring{$^2$}{2}}% (fold)
\label{sec:cnn2}
	\subsection{CNN\texorpdfstring{$^2$}{2} training}% (fold)
	\label{sub:cnn2_training}
		The training, validation, and testing sample size are in Table \ref{tab:CNNsq_sample_size}.
		\begin{table}[htpb]
			\centering
			\caption{The entries in the sum correspond to the $(W^{+}, W^{-}, Z)$ samples.}
			\label{tab:CNNsq_sample_size}
			\begin{tabular}{c|c|c|c}
			Case & Training set     & Validation set & Testing set   \\ \hline
			1    & $11k+12k+10k$    & $2k+2k+2k$     & $3k+3k+3k$    \\
			2    & $80k+84k+72k$    & $19k+21k+18k$  & $25k+26k+22k$ \\
			3    & $194k+205k+176k$ & $48k+51k+44k$  & $60k+64k+55k$
			\end{tabular}
		\end{table}

	% subsection cnn2_training (end)
	\subsection{CNN\texorpdfstring{$^2$}{2} results}% (fold)
	\label{sub:cnn2_results}
		The training results are summarized in Table \ref{tab:CNNsq_training_result}.
		\begin{table}[htpb]
			\centering
			\caption{The training results of $\text{CNN}^2$ for different $\kappa$ samples.}
			\label{tab:CNNsq_training_result}
			\begin{tabular}{c|c|c|cc|cc|cc}
								  &					  &             & \multicolumn{2}{c|}{$W^{+}$} & \multicolumn{2}{c|}{$W^{-}$} & \multicolumn{2}{c}{$Z$} \\
								  & Sample			  & Overall ACC & AUC        & ACC       & AUC        & ACC       & AUC       & ACC       \\ \hline
			$\text{CNN}^2$ $\kappa=0.15$ & \multirow{2}{*}{1}& 0.642 & 0.828 & 0.773 & 0.831 & 0.773 & 0.801 & 0.780 \\
			$\text{CNN}^2$ $\kappa=0.30$ &					 & 0.645 & 0.823 & 0.773 & 0.828 & 0.773 & 0.802 & 0.784 \\ \hline
			$\text{CNN}^2$ $\kappa=0.15$ & \multirow{2}{*}{2}& 0.677 & 0.848 & 0.792 & 0.848 & 0.786 & 0.842 & 0.806 \\
			$\text{CNN}^2$ $\kappa=0.30$ &					 & 0.669 & 0.842 & 0.788 & 0.843 & 0.780 & 0.839 & 0.804 \\ \hline
			$\text{CNN}^2$ $\kappa=0.15$ & \multirow{2}{*}{3}& 0.674 & 0.848 & 0.792 & 0.849 & 0.785 & 0.843 & 0.807 \\
			$\text{CNN}^2$ $\kappa=0.30$ &					 & 0.675 & 0.846 & 0.790 & 0.847 & 0.784 & 0.845 & 0.808 \\
			\end{tabular}
		\end{table}	
		
	% subsection cnn2_results (end)	
% section cnn2 (end)		
\section{Modify preprocess and preselection}% (fold)
\label{sec:modify_preprocess_and_preselection}
	Preprocess: For the vector boson jet which $\phi$ coordinate across the $\phi = \pm\pi$ boundary, plus its $\phi$ coordinate by  $\pi$ such that its center is located around $\phi = 0$.

	Preselection: Include the events with only one jet that can pass the cuts.

	\subsection{CNN results for modified preprocess and preselection}% (fold)
	\label{sub:cnn_results_for_modified_preprocess_and_preselection}
		The training, validation, and testing sample size are in Table \ref{tab:CNN_sample_size_modified}.
		\begin{table}[htpb]
			\centering
			\caption{The entries in the sum correspond to the $(W^{+}, W^{-}, Z)$ samples.}
			\label{tab:CNN_sample_size_modified}
			\begin{tabular}{c|c|c|c}
			Case & Training set     & Validation set & Testing set   \\ \hline
			1    & $313k+323k+307k$ & $78k+80k+76k$  & $97k+100k+96k$ \\
			\end{tabular}
		\end{table}

		The training results are summarized in Table \ref{tab:CNN_training_result_modified}.
		\begin{table}[htpb]
			\centering
			\caption{The training results of CNN with modified preprocessing and preselection.}
			\label{tab:CNN_training_result_modified}
			\begin{tabular}{c|c|c|cc|cc|cc}
								  &					  &             & \multicolumn{2}{c|}{$W^{+}$} & \multicolumn{2}{c|}{$W^{-}$} & \multicolumn{2}{c}{$Z$} \\
								  & Sample			  & Overall ACC & AUC        & ACC       & AUC        & ACC       & AUC       & ACC       \\ \hline
				CNN $\kappa=0.15$ & \multirow{1}{*}{1}& 0.684 & 0.857 & 0.798 & 0.855 & 0.794 & 0.828 & 0.793\\
				CNN${}^2$ $\kappa=0.15$ & \multirow{1}{*}{1}& 0.684 & 0.855 & 0.797 & 0.854 & 0.793 & 0.835 & 0.797\\
			\end{tabular}
		\end{table}	

	% subsection cnn_results_for_modified_preprocess_and_preselection (end)
	\subsection{CNN results for Fishbone}% (fold)
	\label{sub:cnn_results_for_fishbone}
		The training, validation, and testing sample size are in Table \ref{tab:CNN_sample_size_fish}. For case 1, only consider the events with 2 jets that can pass the cuts. For case 2, the events with only 1 jet can pass the cuts are included. 
		\begin{table}[htpb]
			\centering
			\caption{The entries in the sum correspond to the $(W^{+}, W^{-}, Z)$ samples.}
			\label{tab:CNN_sample_size_fish}
			\begin{tabular}{c|c|c|c}
			Case & Training set     & Validation set & Testing set   \\ \hline
			1    & $61k+65k+102k$ & $15k+16k+25k$ & $19k+20k+32k$\\
			2    & $313k+323k+307k$ & $78k+81k+76k$ & $97k+101k+95k$ \\
			\end{tabular}
		\end{table}
		The training results are summarized in Table \ref{tab:CNN_training_result_fish}.
		\begin{table}[htpb]
			\centering
			\caption{The training results of CNN with Fishbone's code.}
			\label{tab:CNN_training_result_fish}
			\begin{tabular}{c|c|c|cc|cc|cc}
								  &					  &             & \multicolumn{2}{c|}{$W^{+}$} & \multicolumn{2}{c|}{$W^{-}$} & \multicolumn{2}{c}{$Z$} \\
								  & Sample			  & Overall ACC & AUC        & ACC       & AUC        & ACC       & AUC       & ACC       \\ \hline
				CNN $\kappa=0.15$ & \multirow{1}{*}{1}& 0.694 & 0.857 & 0.820 & 0.856 & 0.814 & 0.842 & 0.776 \\
				CNN $\kappa=0.15$ & \multirow{1}{*}{2}& 0.682 & 0.856 & 0.797 & 0.854 & 0.793 & 0.827 & 0.791 \\
			\end{tabular}
		\end{table}

	% subsection cnn_results_for_fishbone (end)	
% section modify_preprocess_and_preselection (end)		
\section{Old sample for training}% (fold)
\label{sec:old_sample_for_training}
	This section uses the model \verb+GM_UFO+ and different parameter settings. The model and parameters are the same as the paper. The selection criteria are the same as Sec.\ref{sub:sample_selection}. The events with only one vector boson passing the criteria are included.

	Following are the MadGraph scripts for generating $W^{+}$ sample:
	\begin{verbatim}
	import model GM_UFO
	define v = z w+ w-
	generate p p > H5pp j j $$v, (H5pp > w+ w+, (w+ > j j), (w+ > j j))
	output VBF_H5pp_ww_jjjj-J

	launch VBF_H5pp_ww_jjjj-J

	shower=Pythia8
	detector=Delphes
	analysis=OFF
	done

	set param_card tanth 2.234400e+01
	set param_card lam2 1.040100e+00
	set param_card lam3 8.829540e+00
	set param_card lam4 -2.232270e+00
	set param_card lam5 7.672600e+00
	set param_card M1coeff 1.000000e+02
	set param_card M2coeff 1.000000e+02

	set param_card wt 1.000000e+00
	set param_card wz 1.000000e+00
	set param_card ww 1.000000e+00
	set param_card wh Auto
	set param_card wh__2 Auto
	set param_card wh3p Auto
	set param_card wh3z Auto
	set param_card wh5pp Auto
	set param_card wh5p Auto
	set param_card wh5z Auto

	set run_card nevents 10000
	set run_card ebeam1 6500.0
	set run_card ebeam2 6500.0

	/home/r10222035/boosted_V_ML_test/Cards/delphes_card.dat

	done
	\end{verbatim}

	\subsection{Training results}% (fold)
	\label{sub:training_results}
		The training, validation, and testing sample size are in Table \ref{tab:sample_size_old}.
		\begin{table}[htpb]
			\centering
			\caption{The entries in the sum correspond to the $(W^{+}, W^{-}, Z)$ samples.}
			\label{tab:sample_size_old}
			\begin{tabular}{c|c|c|c}
			Case & Training set     & Validation set & Testing set   \\ \hline
			1    & $112k+116k+101k$ & $27k+29k+25k$ & $34k+36k+31k$ \\
			2    & $224k+232k+201k$ & $56k+58k+50k$ & $69k+72k+63k$ \\
			\end{tabular}
		\end{table}

		The training results are summarized in Table \ref{tab:training_result_old}.
		\begin{table}[htpb]
			\centering
			\caption{The training results of CNN with modified preprocessing and preselection.}
			\label{tab:training_result_old}
			\begin{tabular}{c|c|c|cc|cc|cc}
										&					  &             & \multicolumn{2}{c|}{$W^{+}$} & \multicolumn{2}{c|}{$W^{-}$} & \multicolumn{2}{c}{$Z$} \\
										& Sample			  & Overall ACC & AUC        & ACC       & AUC        & ACC       & AUC       & ACC       \\ \hline
				CNN $\kappa=0.15$		& \multirow{2}{*}{1}  & 0.690 & 0.861 & 0.798 & 0.858 & 0.793 & 0.830 & 0.808 \\
				CNN${}^2$ $\kappa=0.15$ &					  & 0.693 & 0.860 & 0.797 & 0.859 & 0.794 & 0.838 & 0.816 \\ \hline
				CNN $\kappa=0.15$		& \multirow{2}{*}{2}  & 0.697 & 0.864 & 0.801 & 0.863 & 0.796 & 0.837 & 0.812 \\
				CNN${}^2$ $\kappa=0.15$ &					  & 0.698 & 0.863 & 0.800 & 0.862 & 0.797 & 0.844 & 0.818 \\ \hline
			\end{tabular}
		\end{table}
		The results are better than the Sec.\ref{sub:cnn_results_for_modified_preprocess_and_preselection}.
	% subsection training_results (end)

% section old_sample_for_training (end)		
\section{Full event sample}% (fold)
\label{sec:full_event_sample}
	For the full event case, there are six processes: $H_5^{\pm\pm} \to W^{\pm}W^{\pm} \to jjjj$, $H_5^{\pm} \to W^{\pm}Z \to jjjj$, $H_5\to ZZ \to jjjj$, $H_5\to W^{+}W^{-} \to jjjj$.

	\subsection{Full event sample selection}% (fold)
	\label{sub:full_event_sample_selection}	
		Only the events that satisfy the following requirements will be used for training.
		\begin{enumerate}
			\item The transverse momentum of vector boson jets are required $p_\text{T} \in (350, 450) \text{ GeV}$ and in range $\abs{\eta}<1$.
			\item Merging: The angular distance between the two quarks decayed from the vector boson is required $\Delta R(q_1,q_2) < 0.6$.
			\item Matching: The vector boson and jet are matched if $\Delta R(V,j) < 0.1$. The events with less than two matching jets will be discarded.
		\end{enumerate}
	
	% subsection full_event_sample_selection (end)
	\subsection{Full event sample pre-processing}% (fold)
	\label{sub:full_event_sample_pre_processing}
		\subsubsection{Event image}% (fold)
		\label{subs:event_image}
			After the sample selection, we can construct the event image. The event image contains 2 vector boson jets and 2 forward jets. The vector boson jets are selected by the merging requirement. The forward jets are selected by the highest $p_\text{T}$ jets from the remaining jets. The event image is constructed by the following steps:
			\begin{enumerate}
				\item Centralization: Calculating the $p_\text{T}$ weighted center in $\phi$ coordinate, then shift this point to origin.
				\item Flipping: Flip the highest intensity quadrant to the first quadrant. 
				\item Pixelating in $\Delta\eta = \Delta\phi = 6$ box, with $75\times 75$ pixels.
			\end{enumerate}
		% subsubsection event_image (end)	
	% subsection full_event_sample_pre_processing (end)	
	\subsection{Event sample plots}% (fold)
	\label{sub:event_sample_plots}
		Figure \ref{fig:single_event_image_PT}, \ref{fig:single_event_image_Qk} are the single event images of $p_\text{T}$ and $\mathcal{Q}_\kappa$. Figure \ref{fig:event_image_PT}, \ref{fig:event_image_Qk} are the average event images of $p_\text{T}$ and $\mathcal{Q}_\kappa$.
		\begin{figure}[htpb]
			\centering
			\includegraphics[width=0.95\textwidth]{single_event_image_PT.png}
			\caption{Single event images in the $p_\text{T}$ channel.}
			\label{fig:single_event_image_PT}
		\end{figure}
		\begin{figure}[htpb]
			\centering
			\includegraphics[width=0.95\textwidth]{single_event_image_Qk.png}
			\caption{Single event images in the $\mathcal{Q}_\kappa$ channel, with $\kappa = 0.15$.}
			\label{fig:single_event_image_Qk}
		\end{figure}
		\begin{figure}[htpb]
			\centering
			\includegraphics[width=0.95\textwidth]{event_image_PT.png}
			\caption{Average of event images in the $p_\text{T}$ channel.}
			\label{fig:event_image_PT}
		\end{figure}
		\begin{figure}[htpb]
			\centering
			\includegraphics[width=0.95\textwidth]{event_image_Qk.png}
			\caption{Average of event images in the $\mathcal{Q}_\kappa$ channel, with $\kappa = 0.15$.}
			\label{fig:event_image_Qk}
		\end{figure}

		Figure \ref{fig:event_image_PT_ZZ-W+W+} is the $ZZ$ event image minus $W^{+}W^{+}$ event image in $p_\text{T}$ channel.
		\begin{figure}[htpb]
			\centering
			\includegraphics[width=0.5\textwidth]{event_image_PT_ZZ-W+W+.png}
			\caption{The difference between the $ZZ$ and $W^{+}W^{+}$ average event images in $p_\text{T}$ channel.}
			\label{fig:event_image_PT_ZZ-W+W+}
		\end{figure}

	% subsection event_sample_plots (end)		
% section full_event_sample (end)		
\section{Correct decay width sample}% (fold)
\label{sec:correct_decay_width_sample}
	In this section, the samples are generated with the correct decay widths, i.e., the decay width of $t, W$, and $Z$ do not change and the exotic Higgses in the GM model are set to “Auto”, meaning the decay widths are calculated by MadGraph.

	Following are the MadGraph scripts for generating $W^{+}$ sample:
	\begin{verbatim}
	import model GM_UFO
	define v = z w+ w-
	generate p p > H5pp j j $$v, (H5pp > w+ w+, (w+ > j j), (w+ > j j))
	output VBF_H5pp_ww_jjjj

	launch VBF_H5pp_ww_jjjj

	shower=Pythia8
	detector=Delphes
	analysis=OFF
	done

	set param_card tanth 2.234400e+01
	set param_card lam2 1.040100e+00
	set param_card lam3 8.829540e+00
	set param_card lam4 -2.232270e+00
	set param_card lam5 7.672600e+00
	set param_card M1coeff 1.000000e+02
	set param_card M2coeff 1.000000e+02

	set param_card wh Auto
	set param_card wh__2 Auto
	set param_card wh3p Auto
	set param_card wh3z Auto
	set param_card wh5pp Auto
	set param_card wh5p Auto
	set param_card wh5z Auto

	set run_card nevents 10000
	set run_card ebeam1 6500.0
	set run_card ebeam2 6500.0

	/home/r10222035/boosted_V_ML_test/Cards/delphes_card.dat

	done
	\end{verbatim}
	\subsection{Training results}% (fold)
	\label{sub:training_results_correct_decay_width}
		The training, validation, and testing sample size are in Table \ref{tab:sample_size_correct_decay_width}.
		\begin{table}[htpb]
			\centering
			\caption{The entries in the sum correspond to the $(W^{+}, W^{-}, Z)$ samples.}
			\label{tab:sample_size_correct_decay_width}
			\begin{tabular}{c|c|c|c}
			Case & Training set     & Validation set & Testing set   \\ \hline
			1    & $223k+232k+202k$ & $56k+57k+50k$ & $69k+72k+62k$\\
			\end{tabular}
		\end{table}

		The training results are summarized in Table \ref{tab:training_result_correct_decay_width}.
		\begin{table}[htpb]
			\centering
			\caption{The training results of correct decay width sample. The training results of CNN and CNN$^2$ are presented with an average and a standard deviation. These values are derived from 10 times training with the same dataset. Yet, the ACC value of each boson is only derived from a single result.}
			\label{tab:training_result_correct_decay_width}
			\resizebox{\textwidth}{!}{
			\begin{tabular}{c|c|c|cc|cc|cc}
										&					  &             & \multicolumn{2}{c|}{$W^{+}$} & \multicolumn{2}{c|}{$W^{-}$} & \multicolumn{2}{c}{$Z$} \\
										& Sample			  & Overall ACC & AUC        & ACC       & AUC        & ACC       & AUC       & ACC       \\ \hline
				BDT $\kappa=0.30$       &\multirow{3}{*}{1}    & 65.1 & 82.6 & 77.5 & 82.5 & 77.0 & 79.2 & 77.4\\
				CNN $\kappa=0.15$       &					   & 68.24$\pm$0.04 & 85.72$\pm$0.02 & (79.5) & 85.58$\pm$0.02 & (79.1) & 81.64$\pm$0.05 & (79.8)\\
				CNN$^2$ $\kappa=0.15$   &					   & 68.26$\pm$0.13 & 85.55$\pm$0.04 & (79.5) & 85.41$\pm$0.05 & (79.0) & 82.22$\pm$0.08 & (80.2)\\
			\end{tabular}
			}
		\end{table}
		The results are similar to the Sec.\ref{sub:cnn_results_for_modified_preprocess_and_preselection} but worse than Sec.\ref{sub:training_results}. It seems that the decay width of $t, W$, and  $Z$ will affect the training results.

		Figure \ref{fig:CNN learning curve_correct_decay_width} is CNN's loss and accuracy curve. Figure \ref{fig:CNN roc curve_correct_decay_width} is CNN's ROC.
		\begin{figure}[htpb]
			\centering
			\includegraphics[width=0.90\textwidth]{CNN_loss_and_accuracy_correct_width.png}
			\caption{The loss and accuracy curve for the CNN model. Both of them are demonstrated with the average value (solid curve) and the first standard deviation range (error bar).}
			\label{fig:CNN learning curve_correct_decay_width}
		\end{figure}
		\begin{figure}[htpb]
			\centering
			\includegraphics[width=0.95\textwidth]{CNN_roc_auc_correct_width.png}
			\caption{The ROC curve of each vector boson for the CNN model. The plotting scheme is the same as Figure \ref{fig:CNN learning curve_correct_decay_width}.}
			\label{fig:CNN roc curve_correct_decay_width}
		\end{figure}
		
		Figure \ref{fig:CNNsq learning curve_correct_decay_width} is CNN${}^2$'s loss and accuracy curve. Figure \ref{fig:CNNsq roc curve_correct_decay_width} is CNN${}^2$'s ROC.
		\begin{figure}[htpb]
			\centering
			\includegraphics[width=0.90\textwidth]{CNNsq_loss_and_accuracy_correct_width.png}
			\caption{The loss and accuracy curve for the CNN$^2$ model. The plotting scheme is the same as Figure \ref{fig:CNN learning curve_correct_decay_width}.}
			\label{fig:CNNsq learning curve_correct_decay_width}
		\end{figure}
		\begin{figure}[htpb]
			\centering
			\includegraphics[width=0.95\textwidth]{CNNsq_roc_auc_correct_width.png}
			\caption{The ROC curve of each vector boson for the CNN$^2$ model. The plotting scheme is the same as Figure \ref{fig:CNN learning curve_correct_decay_width}.}
			\label{fig:CNNsq roc curve_correct_decay_width}
		\end{figure}

		\begin{figure}[htpb]
			\centering
			\subfloat[Paper]{
				\includegraphics[width=0.4\textwidth]{ROC-paper-wm.png}
			}
			\subfloat[Ours]{
				\includegraphics[width=0.4\textwidth]{ROC_kappa0.15-1000k-pstyle-wm.png}
			}
			\caption{The signal $W^{-}$ ROCs. Left is paper. Right is my result.}
			\label{fig:ROCs_wm}
		\end{figure}
		\begin{figure}[htpb]
			\centering
			\subfloat[Paper]{
				\includegraphics[width=0.4\textwidth]{ROC-paper-z.png}
			}
			\subfloat[Ours]{
				\includegraphics[width=0.4\textwidth]{ROC_kappa0.15-1000k-pstyle-z.png}
			}
			\caption{The signal $Z$ ROCs. Left is paper. Right is my result.}
			\label{fig:ROCs_z}
		\end{figure}

		\begin{figure}[htpb]
			\centering
			\subfloat[CNN]{
				\includegraphics[width=0.4\textwidth]{ROC_CNN_kappa0.15-1000k-pstyle.png}
			}
			\subfloat[CNN${}^2$]{
				\includegraphics[width=0.4\textwidth]{ROC_CNNsq_kappa0.15-1000k-pstyle.png}
			}
			\caption{The ROCs of my training results. For the left one, the model is CNN. For right one, the model is CNN${}^{2}$.}
			\label{fig:ROCs_my}
		\end{figure}

	% subsection training_results (end)
% section correct_decay_width_sample (end)		
\end{document} 
